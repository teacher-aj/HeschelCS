\documentclass[10pt]{article}

\usepackage[margin=1in]{geometry}
\usepackage{amsmath}
\usepackage{amssymb}
\usepackage{amsthm}
\usepackage{mathtools}
\usepackage[shortlabels]{enumitem}
\usepackage[normalem]{ulem}
\usepackage{courier}

\usepackage{hyperref}
\hypersetup{
  colorlinks   = true, %Colours links instead of ugly boxes
  urlcolor     = black, %Colour for external hyperlinks
  linkcolor    = blue, %Colour of internal links
  citecolor    = blue  %Colour of citations
}

\usepackage[T1]{fontenc}
\usepackage{listings}
\lstset{
    language=HTML
    ,basicstyle=\ttfamily
    %,numbers=left
    ,breaklines=true
    }

%%%%%%%%%%%%%%%%%%%%%%%%%%%%%%%%%%%%%%%%%%%%%%%%%%%%%%%%%%%%%%%%%%%%%%%%%%%%%%%%

\theoremstyle{definition}
\newtheorem{problem}{Problem}
\newcommand{\E}{\mathbb E}
\newcommand{\R}{\mathbb R}
\DeclareMathOperator{\Var}{Var}
\DeclareMathOperator*{\argmin}{arg\,min}
\DeclareMathOperator*{\argmax}{arg\,max}

\newcommand{\trans}[1]{{#1}^{T}}
\newcommand{\loss}{\ell}
\newcommand{\w}{\mathbf w}
\newcommand{\mle}[1]{\hat{#1}_{\textit{mle}}}
\newcommand{\map}[1]{\hat{#1}_{\textit{map}}}
\newcommand{\normal}{\mathcal{N}}
\newcommand{\x}{\mathbf x}
\newcommand{\y}{\mathbf y}
\newcommand{\ltwo}[1]{\lVert {#1} \rVert}

%%%%%%%%%%%%%%%%%%%%%%%%%%%%%%%%%%%%%%%%%%%%%%%%%%%%%%%%%%%%%%%%%%%%%%%%%%%%%%%%

\begin{document}
\begin{center}
    {
\Large
Quiz: Exceptions
}

    \vspace{0.1in}
    CSCI040: \sout{Computing for the Web} Introduction to Hacking

    \vspace{0.1in}
\end{center}

\vspace{0.15in}
\noindent
\textbf{Total Score:} ~~~~~~~~~~~~~~~/5

\vspace{0.5in}
\noindent
\textbf{Printed Name:}

\noindent
\rule{\textwidth}{0.1pt}
\vspace{0.25in}

\noindent
\textbf{Quiz rules:}
\begin{enumerate}
    \item You MAY use any printed or handwritten notes.
    \item You MAY NOT use a computer or any other electronic device.
\end{enumerate}

\noindent

\vspace{0.15in}


%\begin{problem}
    %The following code (circle one)
%
    %\vspace{0.25in}
    %\hspace{0.5in}terminates without error 
    %\hspace{1in}throws an exception
    %\hspace{1in}runs forever
    %\vspace{0.25in}
%
    %\noindent
    %If the code terminates without error, write the output.
    %If the code throws an exception, state the exception.
%\end{problem}
%\begin{lstlisting}
%xs = [1, 2, 3]
%total = 0
%while xs:
    %total //= total
%print('len(xs)=',len(xs))
%\end{lstlisting}
%\vspace{1in}


\begin{problem}
    The following code (circle one)

    \vspace{0.25in}
    \hspace{0.5in}terminates without error 
    \hspace{1in}throws an exception
    \hspace{1in}runs forever
    \vspace{0.25in}

    \noindent
    If the code terminates without error, write the output.
    If the code throws an exception, state the exception.
\end{problem}
\begin{lstlisting}
grades={
    'alice':{'hw1':99,'hw2':88},
    'bob':{'hw1':82,'hw2':91},
}
alice_points = 0
for v in grades[alice].values():
    alice_points += grades[v]
print('alice_points=', alice_points)
\end{lstlisting}
\vspace{0.75in}


\begin{problem}
    The following code (circle one)

    \vspace{0.25in}
    \hspace{0.5in}terminates without error 
    \hspace{1in}throws an exception
    \hspace{1in}runs forever
    \vspace{0.25in}

    \noindent
    If the code terminates without error, write the output.
    If the code throws an exception, state the exception.
\end{problem}
\begin{lstlisting}
grades={
    'alice':{'hw1':99,'hw2':88},
    'bob':{'hw1':82,'hw2':91},
}
output = "grade=" + grades['charlie'][hw1]
print('output=', output)
\end{lstlisting}
\vspace{0.75in}

\begin{problem}
    The following code (circle one)

    \vspace{0.25in}
    \hspace{0.5in}terminates without error 
    \hspace{1in}throws an exception
    \hspace{1in}runs forever
    \vspace{0.25in}

    \noindent
    If the code terminates without error, write the output.
    If the code throws an exception, state the exception.
\end{problem}
\begin{lstlisting}
xs = [1, 2, 3]
try:
    result = xs[3]
    result = 5
except IndexError:
    result = -1
print('result=', result)
\end{lstlisting}
\vspace{2in}

\begin{problem}
    The following code (circle one)

    \vspace{0.25in}
    \hspace{0.5in}terminates without error 
    \hspace{1in}throws an exception
    \hspace{1in}runs forever
    \vspace{0.25in}

    \noindent
    If the code terminates without error, write the output.
    If the code throws an exception, state the exception.
\end{problem}
\begin{lstlisting}
grades={
    'alice':{'hw1':99,'hw2':88},
    'bob':{'hw1':82,'hw2':91},
}
try:
    output = "grade=" + grades['charlie']['hw1']
except KeyError:
    output = 'oops'
print('output=', output)
\end{lstlisting}
\vspace{0.75in}

\newpage
\begin{problem}
    The following code (circle one)

    \vspace{0.25in}
    \hspace{0.5in}terminates without error 
    \hspace{1in}throws an exception
    \hspace{1in}runs forever
    \vspace{0.25in}

    \noindent
    If the code terminates without error, write the output.
    If the code throws an exception, state the exception.
\end{problem}
\begin{lstlisting}
def bar(xs):
    assert(len(xs) > 0)
    return len(xs)*2

result = 0
try:
    result += bar([1,2,3])
    result += bar([2,3])
    result += bar([])
    result += bar([5])
except AssertionError:
    pass

print('result=', result)
\end{lstlisting}
\vspace{0.75in}
\end{document}


