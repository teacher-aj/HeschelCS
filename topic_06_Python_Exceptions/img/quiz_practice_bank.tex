\documentclass[10pt]{article}

\usepackage[margin=1in]{geometry}
\usepackage{amsmath}
\usepackage{amssymb}
\usepackage{amsthm}
\usepackage{mathtools}
\usepackage[shortlabels]{enumitem}
\usepackage[normalem]{ulem}

\usepackage{hyperref}
\hypersetup{
  colorlinks   = true, %Colours links instead of ugly boxes
  urlcolor     = black, %Colour for external hyperlinks
  linkcolor    = blue, %Colour of internal links
  citecolor    = blue  %Colour of citations
}

\usepackage{listings}
\lstset{language=Python,breaklines=true} %,numbers=left}

%%%%%%%%%%%%%%%%%%%%%%%%%%%%%%%%%%%%%%%%%%%%%%%%%%%%%%%%%%%%%%%%%%%%%%%%%%%%%%%%

\theoremstyle{definition}
\newtheorem{problem}{Problem}
\newcommand{\E}{\mathbb E}
\newcommand{\R}{\mathbb R}
\DeclareMathOperator{\Var}{Var}
\DeclareMathOperator*{\argmin}{arg\,min}
\DeclareMathOperator*{\argmax}{arg\,max}

\newcommand{\trans}[1]{{#1}^{T}}
\newcommand{\loss}{\ell}
\newcommand{\w}{\mathbf w}
\newcommand{\mle}[1]{\hat{#1}_{\textit{mle}}}
\newcommand{\map}[1]{\hat{#1}_{\textit{map}}}
\newcommand{\normal}{\mathcal{N}}
\newcommand{\x}{\mathbf x}
\newcommand{\y}{\mathbf y}
\newcommand{\ltwo}[1]{\lVert {#1} \rVert}

%%%%%%%%%%%%%%%%%%%%%%%%%%%%%%%%%%%%%%%%%%%%%%%%%%%%%%%%%%%%%%%%%%%%%%%%%%%%%%%%

\begin{document}

\begin{center}
    {
\Large
    Exceptions Question Bank
}

    \vspace{0.1in}
    CSCI040: \sout{Computing for the Web} Introduction to Hacking

    \vspace{0.1in}
\end{center}


\noindent
Your exceptions quiz will contain 5 questions that follow the pattern in this question bank.

\vspace{0.25in}
\noindent
Recall that you are responsible for knowing the following exceptions:
\lstinline{AssertionError},
\lstinline{AttributeError},
\lstinline{IndexError},
\lstinline{KeyError},
\lstinline{NameError},
\lstinline{UnboundLocalError},
\lstinline{TypeError},
\lstinline{ZeroDivisionError}.
\vspace{0.25in}

\begin{problem}
    The following code (circle one)

    \vspace{0.25in}
    \hspace{0.5in}terminates without error 
    \hspace{1in}throws an exception
    \hspace{1in}runs forever
    \vspace{0.25in}

    \noindent
    If the code terminates without error, write the output.
    If the code throws an exception, state the exception.
\end{problem}
\begin{lstlisting}
xs = []
total = 0
while xs:
    total += 1
print('total=', total)
\end{lstlisting}
\vspace{1in}


\begin{problem}
    The following code (circle one)

    \vspace{0.25in}
    \hspace{0.5in}terminates without error 
    \hspace{1in}throws an exception
    \hspace{1in}runs forever
    \vspace{0.25in}

    \noindent
    If the code terminates without error, write the output.
    If the code throws an exception, state the exception.
\end{problem}
\begin{lstlisting}
xs = []
while xs:
    total += 1
    assert(xs)
print('total=', total)
\end{lstlisting}
\vspace{1in}


\newpage
\begin{problem}
    The following code (circle one)

    \vspace{0.25in}
    \hspace{0.5in}terminates without error 
    \hspace{1in}throws an exception
    \hspace{1in}runs forever
    \vspace{0.25in}

    \noindent
    If the code terminates without error, write the output.
    If the code throws an exception, state the exception.
\end{problem}
\begin{lstlisting}
xs = []
while len(xs)<5:
    xs.append('test')
print('len(xs)=',len(xs))
\end{lstlisting}
\vspace{1in}


\begin{problem}
    The following code (circle one)

    \vspace{0.25in}
    \hspace{0.5in}terminates without error 
    \hspace{1in}throws an exception
    \hspace{1in}runs forever
    \vspace{0.25in}

    \noindent
    If the code terminates without error, write the output.
    If the code throws an exception, state the exception.
\end{problem}
\begin{lstlisting}
xs = []
while len(xs)<5:
    xs.append('test')
assert(xs)
print('len(xs)=',len(xs))
\end{lstlisting}
\vspace{1in}


\begin{problem}
    The following code (circle one)

    \vspace{0.25in}
    \hspace{0.5in}terminates without error 
    \hspace{1in}throws an exception
    \hspace{1in}runs forever
    \vspace{0.25in}

    \noindent
    If the code terminates without error, write the output.
    If the code throws an exception, state the exception.
\end{problem}
\begin{lstlisting}
xs = [1, 2, 3]
while xs:
    xs.append('test')
print('len(xs)=',len(xs))
\end{lstlisting}
\vspace{1in}

\begin{problem}
    The following code (circle one)

    \vspace{0.25in}
    \hspace{0.5in}terminates without error 
    \hspace{1in}throws an exception
    \hspace{1in}runs forever
    \vspace{0.25in}

    \noindent
    If the code terminates without error, write the output.
    If the code throws an exception, state the exception.
\end{problem}
\begin{lstlisting}
xs = [1, 2, 3]
total = 0
while xs:
    total //= total
print('len(xs)=',len(xs))
\end{lstlisting}
\vspace{1in}

\begin{problem}
    The following code (circle one)

    \vspace{0.25in}
    \hspace{0.5in}terminates without error 
    \hspace{1in}throws an exception
    \hspace{1in}runs forever
    \vspace{0.25in}

    \noindent
    If the code terminates without error, write the output.
    If the code throws an exception, state the exception.
\end{problem}
\begin{lstlisting}
xs = [1, 2, 3]
while xs:
    xs.append('test')
    assert('t' in xs)
print('len(xs)=',len(xs))
\end{lstlisting}
\vspace{1in}

\begin{problem}
    The following code (circle one)

    \vspace{0.25in}
    \hspace{0.5in}terminates without error 
    \hspace{1in}throws an exception
    \hspace{1in}runs forever
    \vspace{0.25in}

    \noindent
    If the code terminates without error, write the output.
    If the code throws an exception, state the exception.
\end{problem}
\begin{lstlisting}
xs = [1, 2, 3]
while xs:
    xs += 'test'
print('len(xs)=',len(xs))
\end{lstlisting}
\vspace{1in}

\begin{problem}
    The following code (circle one)

    \vspace{0.25in}
    \hspace{0.5in}terminates without error 
    \hspace{1in}throws an exception
    \hspace{1in}runs forever
    \vspace{0.25in}

    \noindent
    If the code terminates without error, write the output.
    If the code throws an exception, state the exception.
\end{problem}
\begin{lstlisting}
xs = [1, 2, 3]
total = 10
for x in xs:
    total %= x
print("total=",total)
\end{lstlisting}
\vspace{1in}

\begin{problem}
    The following code (circle one)

    \vspace{0.25in}
    \hspace{0.5in}terminates without error 
    \hspace{1in}throws an exception
    \hspace{1in}runs forever
    \vspace{0.25in}

    \noindent
    If the code terminates without error, write the output.
    If the code throws an exception, state the exception.
\end{problem}
\begin{lstlisting}
xs = [0, 1, 2]
xs.replace(1, 2)
print("xs=", xs)
\end{lstlisting}
\vspace{1in}


\begin{problem}
    The following code (circle one)

    \vspace{0.25in}
    \hspace{0.5in}terminates without error 
    \hspace{1in}throws an exception
    \hspace{1in}runs forever
    \vspace{0.25in}

    \noindent
    If the code terminates without error, write the output.
    If the code throws an exception, state the exception.
\end{problem}
\begin{lstlisting}
s = 'hello world'
i = s.find(' ')
print("i=", i)
\end{lstlisting}
\vspace{1in}


\newpage
\begin{problem}
    The following code (circle one)

    \vspace{0.25in}
    \hspace{0.5in}terminates without error 
    \hspace{1in}throws an exception
    \hspace{1in}runs forever
    \vspace{0.25in}

    \noindent
    If the code terminates without error, write the output.
    If the code throws an exception, state the exception.
\end{problem}
\begin{lstlisting}
grades={
    'alice':{'hw1':99,'hw2':88},
    'bob':{'hw1':82,'hw2':91},
}
for k,v in sorted(grades.items()):
    print(v['hw1'])
\end{lstlisting}
\vspace{0.75in}


\begin{problem}
    The following code (circle one)

    \vspace{0.25in}
    \hspace{0.5in}terminates without error 
    \hspace{1in}throws an exception
    \hspace{1in}runs forever
    \vspace{0.25in}

    \noindent
    If the code terminates without error, write the output.
    If the code throws an exception, state the exception.
\end{problem}
\begin{lstlisting}
grades={
    'alice':{'hw1':99,'hw2':88},
    'bob':{'hw1':82,'hw2':91},
}
for k,v in sorted(grades.items()):
    print(k['hw1'])
\end{lstlisting}
\vspace{0.75in}

\begin{problem}
    The following code (circle one)

    \vspace{0.25in}
    \hspace{0.5in}terminates without error 
    \hspace{1in}throws an exception
    \hspace{1in}runs forever
    \vspace{0.25in}

    \noindent
    If the code terminates without error, write the output.
    If the code throws an exception, state the exception.
\end{problem}
\begin{lstlisting}
grades={
    'alice':{'hw1':99,'hw2':88},
    'bob':{'hw1':82,'hw2':91},
}
for k,v in sorted(grades.items()):
    print(v[0])
\end{lstlisting}
\vspace{0.75in}


\begin{problem}
    The following code (circle one)

    \vspace{0.25in}
    \hspace{0.5in}terminates without error 
    \hspace{1in}throws an exception
    \hspace{1in}runs forever
    \vspace{0.25in}

    \noindent
    If the code terminates without error, write the output.
    If the code throws an exception, state the exception.
\end{problem}
\begin{lstlisting}
grades={
    'alice':{'hw1':99,'hw2':88},
    'bob':{'hw1':82,'hw2':91},
}
for k,v in sorted(grades.items()):
    print(k[0])
\end{lstlisting}
\vspace{0.75in}


\begin{problem}
    The following code (circle one)

    \vspace{0.25in}
    \hspace{0.5in}terminates without error 
    \hspace{1in}throws an exception
    \hspace{1in}runs forever
    \vspace{0.25in}

    \noindent
    If the code terminates without error, write the output.
    If the code throws an exception, state the exception.
\end{problem}
\begin{lstlisting}
grades={
    'alice':{'hw1':99,'hw2':88},
    'bob':{'hw1':82,'hw2':91},
}
output = "grade=" + grades['alice']['hw1']
print('output=', output)
\end{lstlisting}
\vspace{0.75in}


\begin{problem}
    The following code (circle one)

    \vspace{0.25in}
    \hspace{0.5in}terminates without error 
    \hspace{1in}throws an exception
    \hspace{1in}runs forever
    \vspace{0.25in}

    \noindent
    If the code terminates without error, write the output.
    If the code throws an exception, state the exception.
\end{problem}
\begin{lstlisting}
grades={
    'alice':{'hw1':99,'hw2':88},
    'bob':{'hw1':82,'hw2':91},
}
output = "grade=" + grades['charlie']['hw1']
print('output=', output)
\end{lstlisting}
\vspace{0.75in}


\begin{problem}
    The following code (circle one)

    \vspace{0.25in}
    \hspace{0.5in}terminates without error 
    \hspace{1in}throws an exception
    \hspace{1in}runs forever
    \vspace{0.25in}

    \noindent
    If the code terminates without error, write the output.
    If the code throws an exception, state the exception.
\end{problem}
\begin{lstlisting}
grades={
    'alice':{'hw1':99,'hw2':88},
    'bob':{'hw1':82,'hw2':91},
}
output = "grade=" + grades['bob']['hw2'][91]
print('output=', output)
\end{lstlisting}
\vspace{0.75in}


\begin{problem}
    The following code (circle one)

    \vspace{0.25in}
    \hspace{0.5in}terminates without error 
    \hspace{1in}throws an exception
    \hspace{1in}runs forever
    \vspace{0.25in}

    \noindent
    If the code terminates without error, write the output.
    If the code throws an exception, state the exception.
\end{problem}
\begin{lstlisting}
grades={
    'alice':{'hw1':99,'hw2':88},
    'bob':{'hw1':82,'hw2':91},
}
output = alice['hw1']
print('output=', output)
\end{lstlisting}
\vspace{0.75in}


\begin{problem}
    The following code (circle one)

    \vspace{0.25in}
    \hspace{0.5in}terminates without error 
    \hspace{1in}throws an exception
    \hspace{1in}runs forever
    \vspace{0.25in}

    \noindent
    If the code terminates without error, write the output.
    If the code throws an exception, state the exception.
\end{problem}
\begin{lstlisting}
grades={
    'alice':{'hw1':99,'hw2':88},
    'bob':{'hw1':82,'hw2':91},
}
total = 0
for i in grades:
    for j in i:
        total += 1
print('total=',total)
\end{lstlisting}
\vspace{0.75in}


\begin{problem}
    The following code (circle one)

    \vspace{0.25in}
    \hspace{0.5in}terminates without error 
    \hspace{1in}throws an exception
    \hspace{1in}runs forever
    \vspace{0.25in}

    \noindent
    If the code terminates without error, write the output.
    If the code throws an exception, state the exception.
\end{problem}
\begin{lstlisting}
xs = [1, 2, 3]
try:
    result = xs[3]
except IndexError:
    result = -1
print('result=', result)
\end{lstlisting}
\vspace{1in}

\begin{problem}
    The following code (circle one)

    \vspace{0.25in}
    \hspace{0.5in}terminates without error 
    \hspace{1in}throws an exception
    \hspace{1in}runs forever
    \vspace{0.25in}

    \noindent
    If the code terminates without error, write the output.
    If the code throws an exception, state the exception.
\end{problem}
\begin{lstlisting}
xs = [1, 2, 3]
try:
    result = xs[3]
except:
    result = -1
print('result=', result)
\end{lstlisting}
\vspace{1in}


\begin{problem}
    The following code (circle one)

    \vspace{0.25in}
    \hspace{0.5in}terminates without error 
    \hspace{1in}throws an exception
    \hspace{1in}runs forever
    \vspace{0.25in}

    \noindent
    If the code terminates without error, write the output.
    If the code throws an exception, state the exception.
\end{problem}
\begin{lstlisting}
xs = [1, 2, 3]
try:
    result = xs[3]
except NameError:
    result = -1
print('result=', result)
\end{lstlisting}
\vspace{1in}


\begin{problem}
    The following code (circle one)

    \vspace{0.25in}
    \hspace{0.5in}terminates without error 
    \hspace{1in}throws an exception
    \hspace{1in}runs forever
    \vspace{0.25in}

    \noindent
    If the code terminates without error, write the output.
    If the code throws an exception, state the exception.
\end{problem}
\begin{lstlisting}
grades={
    'alice':{'hw1':99,'hw2':88},
    'bob':{'hw1':82,'hw2':91},
}
try:
    output = "grade=" + grades['charlie']['hw1']
except KeyError:
    output = 'oops'
print('output=', output)
\end{lstlisting}
\vspace{0.75in}


\begin{problem}
    The following code (circle one)

    \vspace{0.25in}
    \hspace{0.5in}terminates without error 
    \hspace{1in}throws an exception
    \hspace{1in}runs forever
    \vspace{0.25in}

    \noindent
    If the code terminates without error, write the output.
    If the code throws an exception, state the exception.
\end{problem}
\begin{lstlisting}
grades={
    'alice':{'hw1':99,'hw2':88},
    'bob':{'hw1':82,'hw2':91},
}
try:
    output = "grade=" + grades['charlie']['hw1']
except IndexError:
    output = 'oops'
print('output=', output)
\end{lstlisting}
\vspace{0.75in}

\begin{problem}
    The following code (circle one)

    \vspace{0.25in}
    \hspace{0.5in}terminates without error 
    \hspace{1in}throws an exception
    \hspace{1in}runs forever
    \vspace{0.25in}

    \noindent
    If the code terminates without error, write the output.
    If the code throws an exception, state the exception.
\end{problem}
\begin{lstlisting}
def foo(xs):
    assert(len(xs) > 0)
foo()
\end{lstlisting}
\vspace{0.75in}

\begin{problem}
    The following code (circle one)

    \vspace{0.25in}
    \hspace{0.5in}terminates without error 
    \hspace{1in}throws an exception
    \hspace{1in}runs forever
    \vspace{0.25in}

    \noindent
    If the code terminates without error, write the output.
    If the code throws an exception, state the exception.
\end{problem}
\begin{lstlisting}
def foo(xs):
    assert(len(xs) > 0)

try:
    foo([1,2,3])
except AssertionError:
    pass
\end{lstlisting}
\vspace{0.5in}

\begin{problem}
    The following code (circle one)

    \vspace{0.25in}
    \hspace{0.5in}terminates without error 
    \hspace{1in}throws an exception
    \hspace{1in}runs forever
    \vspace{0.25in}

    \noindent
    If the code terminates without error, write the output.
    If the code throws an exception, state the exception.
\end{problem}
\begin{lstlisting}
def foo(xs):
    assert(len(xs) > 0)

result = 0
try:
    result += foo([1,2,3])
except AssertionError:
    result -= 1
print('result=', result)
\end{lstlisting}
\vspace{0.5in}

\begin{problem}
    The following code (circle one)

    \vspace{0.25in}
    \hspace{0.5in}terminates without error 
    \hspace{1in}throws an exception
    \hspace{1in}runs forever
    \vspace{0.25in}

    \noindent
    If the code terminates without error, write the output.
    If the code throws an exception, state the exception.
\end{problem}
\begin{lstlisting}
def foo(xs):
    assert(len(xs) > 0)

example = 0
try:
    example += foo([1,2,3])
    example += 1
except ValueError:
    pass
print('example=', example)
\end{lstlisting}
\vspace{0.75in}

\begin{problem}
    The following code (circle one)

    \vspace{0.25in}
    \hspace{0.5in}terminates without error 
    \hspace{1in}throws an exception
    \hspace{1in}runs forever
    \vspace{0.25in}

    \noindent
    If the code terminates without error, write the output.
    If the code throws an exception, state the exception.
\end{problem}
\begin{lstlisting}
def foo(xs):
    assert(len(xs) > 0)

example = 0
try:
    example += foo([1,2,3])
    example += 1
except AssertionError:
    pass
print('example=', example)
\end{lstlisting}
\vspace{0.75in}

\begin{problem}
    The following code (circle one)

    \vspace{0.25in}
    \hspace{0.5in}terminates without error 
    \hspace{1in}throws an exception
    \hspace{1in}runs forever
    \vspace{0.25in}

    \noindent
    If the code terminates without error, write the output.
    If the code throws an exception, state the exception.
\end{problem}
\begin{lstlisting}
def bar(xs):
    assert(len(xs) > 0)
    return len(xs)*2

result = 0
try:
    result += bar([1,2,3])
    result += bar([2,3])
    result += bar([])
    result += bar([5])
except AssertionError:
    pass

print('result=', result)
\end{lstlisting}
\vspace{0.75in}

\newpage
\begin{problem}
    The following code (circle one)

    \vspace{0.25in}
    \hspace{0.5in}terminates without error 
    \hspace{1in}throws an exception
    \hspace{1in}runs forever
    \vspace{0.25in}

    \noindent
    If the code terminates without error, write the output.
    If the code throws an exception, state the exception.
\end{problem}
\begin{lstlisting}
def bar(xs):
    assert(len(xs) > 0)
    return len(xs)*2

result = 0
try:
    result += bar([1,2,3])
    result += bar([2,3])
    result += bar([])
    result += bar([5])
except ValueError:
    pass

print('result=', result)
\end{lstlisting}
\vspace{0.75in}

\begin{problem}
    The following code (circle one)

    \vspace{0.25in}
    \hspace{0.5in}terminates without error 
    \hspace{1in}throws an exception
    \hspace{1in}runs forever
    \vspace{0.25in}

    \noindent
    If the code terminates without error, write the output.
    If the code throws an exception, state the exception.
\end{problem}
\begin{lstlisting}
def bar(xs):
    assert(len(xs) > 0)
    return len(xs)*2

result = 0
try:
    result += bar([1,2,3])
    result += bar([2,3])
    result += bar()
    result += bar([5])
except AssertionError:
    result += 1
except TypeError:
    result += 5

print('result=', result)
\end{lstlisting}
\vspace{0.75in}

\newpage
\begin{problem}
    The following code (circle one)

    \vspace{0.25in}
    \hspace{0.5in}terminates without error 
    \hspace{1in}throws an exception
    \hspace{1in}runs forever
    \vspace{0.25in}

    \noindent
    If the code terminates without error, write the output.
    If the code throws an exception, state the exception.
\end{problem}
\begin{lstlisting}
def bar(xs):
    if not len(xs) > 0:
        raise ValueError('input list must be non-empty')
    return len(xs)*2

result = 0
try:
    result += bar([1,2,3])
    result += bar([2,3])
    result += bar()
    result += bar([5])
except AssertionError:
    result += 1
except TypeError:
    result += 5

print('result=', result)
\end{lstlisting}
\vspace{0.75in}

\begin{problem}
    The following code (circle one)

    \vspace{0.25in}
    \hspace{0.5in}terminates without error 
    \hspace{1in}throws an exception
    \hspace{1in}runs forever
    \vspace{0.25in}

    \noindent
    If the code terminates without error, write the output.
    If the code throws an exception, state the exception.
\end{problem}
\begin{lstlisting}
def bar(xs):
    if not len(xs) > 0:
        raise ValueError('input list must be non-empty')
    return len(xs)*2

result = 0
try:
    result += bar([1,2,3])
    result += bar([2,3])
    result += bar()
    result += bar([5])
except ValueError:
    result += 1
except TypeError:
    result += 5

print('result=', result)
\end{lstlisting}
\vspace{0.75in}

\newpage
\begin{problem}
    The following code (circle one)

    \vspace{0.25in}
    \hspace{0.5in}terminates without error 
    \hspace{1in}throws an exception
    \hspace{1in}runs forever
    \vspace{0.25in}

    \noindent
    If the code terminates without error, write the output.
    If the code throws an exception, state the exception.
\end{problem}
\begin{lstlisting}
def bar(xs):
    if not len(xs) > 0:
        raise ValueError('input list must be non-empty')
    return len(xs)*2

result = 0
result += bar

print('result=', result)
\end{lstlisting}
\vspace{0.75in}
\end{document}

